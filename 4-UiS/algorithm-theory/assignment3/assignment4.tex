\documentclass[a4paper, 10pt, twoside]{article}
\usepackage[left=2cm, right=2cm, top=2cm, bottom=3cm]{geometry}
\usepackage{amsmath}
\usepackage[shortlabels]{enumitem}
\usepackage{bbold}
\usepackage{cases}
\usepackage{systeme}
\usepackage{graphicx}
\usepackage{clrscode3e}

\begin{document}

\title{Algorithm Theory - Assignment 4}
\author{T\'eo Bouvard}
\maketitle

\section*{Problem 1}
\begin{enumerate}[a)]
	\item We compute the m-table and the s-table according to the \proc{matrix-chain-order} procedure, with a slight modification. Rather than replacing a value if the computational cost is lower, we replace it when the cost is higher. By doing this, we find the product order which maximizes the number of scalar multiplications.

	      \begin{table}[!htb]
		      \begin{minipage}{.66\textwidth}
			      \centering
			      \caption{m-table}
			      \begin{tabular}{|c|c|c|c|c|}
				      \hline
				      0 & 15750 & 18000 & 21000 & 43875 \\ \hline
				        & 0     & 2625  & 6000  & 17625 \\ \hline
				        &       & 0     & 750   & 4500  \\ \hline
				        &       &       & 0     & 1250  \\ \hline
				        &       &       &       & 0     \\ \hline
			      \end{tabular}
		      \end{minipage}
		      \begin{minipage}{.33\textwidth}
			      \centering
			      \caption{s-table}
			      \begin{tabular}{|c|c|c|c|c|}
				      \hline
				       & 0 & 1 & 1 & 0 \\ \hline
				       &   & 1 & 1 & 1 \\ \hline
				       &   &   & 2 & 3 \\ \hline
				       &   &   &   & 3 \\ \hline
				       &   &   &   &   \\ \hline
			      \end{tabular}
		      \end{minipage}
	      \end{table}
\end{enumerate}


\end{document}
