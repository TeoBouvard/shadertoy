% ========== Config ========== %

\documentclass{article}

\usepackage{fancyhdr} % draw header and footer separations
\usepackage{lastpage} % get last page number
\usepackage[super]{natbib} % format references
\usepackage{url} % pretty wrap url
\usepackage{hyperref} % color url
\hypersetup{urlcolor=blue, colorlinks=true}
\usepackage{pythonhighlight} % python syntax highlighting
\usepackage[left=3cm, right=3cm, bottom=3cm, top=3cm]{geometry} % margin settings


\pagestyle{fancy}
\fancyhead[L]{DAT 510 - Security and Vunlerability in Networks}
\fancyhead[R]{Assignment 1}
\fancyfoot[C]{ \thepage / \pageref{LastPage} }
\fancyfoot[L]{ T\'eo Bouvard }

\renewcommand{\headrulewidth}{0.4pt}
\renewcommand{\footrulewidth}{0.4pt}


% ========== Meta ========== %

\title{\textbf{Vigenere Cipher Cracking and Simplified Data Encryption Standard implementation}}
\author{}
\date{}


% ========== Document ========== %

\begin{document}

\maketitle \thispagestyle{fancy}

\begin{abstract}
    Redact this part later
    
\end{abstract}

\section{Vigenere Cipher Cracking}

    We are given the following cipher :

    \begin{python}
        BQZRMQ KLBOXE WCCEFL DKRYYL BVEHIZ NYJQEE BDYFJO PTLOEM EHOMIC
        UYHHTS GKNJFG EHIMKN IHCTIH VRIHAR SMGQTR QCSXXC SWTNKP TMNSW
        AMXVCY WEOGSR FFUEEB DKQLQZ WRKUCO FTPLOT GOJZRI XEPZSE ISXTCT
        WZRMXI RIHALE SPRFAE FVYORI HNITRG PUHITM CFCDLA HIBKLH RCDIMT
        WQWTOR DJCNDY YWMJCN HDUWOF DPUPNG BANULZ NGYPQU LEUXOV FFDCEE
        YHQUXO YOXQUO DDCVIR RPJCAT RAQVFS AWMJCN HTSOXQ UODDAG BANURR
        REZJGD VJSXOO MSDNIT RGPUHN HRSSSF VFSINH MSGPCM ZJCSLY GEWGQT
        DREASV FPXEAR IMLPZW EHQGMG WSEIXE GQKPRM XIBFWL IPCHYM OTNXYV
        FFDCEE YHASBA TEXCJZ VTSGBA NUDYAP IUGTLD WLKVRI HWACZG PTRYCE
        VNQCUP AOSPEU KPCSNG RIHLRI KUMGFC YTDQES DAHCKP BDUJPX KPYMBD
        IWDQEF WSEVKT CDDWLI NEPZSE OPYIW
    \end{python}

    
    We know that this english message has been encrypted by a polyalphabetic substitution cipher, and that the encryption key is not longer than 10 characters. To break this cipher, the first approach is to try to guess the exact key length. To do so, we are looking for repeating sequences, which have a high probability of corresponding to a repetition of the key. This method is called kasiski examination.\cite{vigenere}

    To perform this examination, I implemented a Python method which loops through the whole cipher, and searches for patterns which appear again in the rest of the cipher. If an occurence of the current pattern is found, the distance between the two patterns is added to the possible keys list. Note that all the factors of 2 of the distance are also added to the list, as the key could have repeated 
    multiple times between the two occurences of the pattern. The get factors submethod does not return factors
    which are greater than the maximum key length.

    \begin{python}

def kasiski_examination(cipher):
''' Sort the key length probabilities by identifying repeated substrings '''

    possible_key_lengths = []

    for pattern_length in range(MIN_PATTERN_LENGTH, MAX_PATTERN_LENGTH):
        for index in range(len(cipher)):
            substring = cipher[index:index+pattern_length]
            distance = cipher[index+pattern_length:].find(substring)

            if distance != -1:
                possible_key_lengths.extend(get_factors(distance + pattern_length))

    probable_key_lengths = {length:possible_key_lengths.count(length) for length in possible_key_lengths}

    return sorted(probable_key_lengths.items(), key=lambda x:x[1], reverse=True)

    \end{python}

When executed on the given cipher with MIN\_PATTERN\_LENGTH = 5 and MAX\_PATTERN\_LENGTH = 10, we get this output.

\section{Simplified DES Implementation}

\section{Conclusion}

% ========== References ========== %

\bibliographystyle{unsrtnat}
\bibliography{sources}

\end{document}



