% ========== Config ========== %

\documentclass{article}

\usepackage{fancyhdr} % draw header and footer separations
\usepackage{lastpage} % get last page number
\usepackage[super]{natbib} % format references
\usepackage{url} % pretty wrap url
\usepackage{indentfirst} % indent every paragraph
\usepackage{hyperref} % color url
\hypersetup{urlcolor=blue, colorlinks=true}
\usepackage{pythonhighlight} % python syntax highlighting
\usepackage{amsmath}
\usepackage[left=2cm, right=2cm, bottom=3cm, top=3cm]{geometry} % margin settings
\usepackage{alltt}
\renewcommand{\ttdefault}{txtt}


\pagestyle{fancy}
\fancyhead[L]{DAT 510 - Security and Vulnerability in Networks}
\fancyhead[R]{Assignment 2}
\fancyfoot[C]{ \thepage / \pageref{LastPage} }
\fancyfoot[L]{ T\'eo Bouvard }

\renewcommand{\headrulewidth}{0.4pt}
\renewcommand{\footrulewidth}{0.4pt}


% ========== Meta ========== %

\title{\textbf{Secure Communication Protocols}}
\author{}
\date{}


% ========== Document ========== %

\begin{document}

\maketitle \thispagestyle{fancy}

\begin{abstract}
    In this assignment, we implement a secure communication protocol consisting of three consecutive steps. First, we use Diffie-Hellman\cite{stallings_2017} key exchange to craft a shared private key between two actors using public key cryptography. We then use an implementation of the Blum Blum Shub algorithm to generate a cryptographically secure pseudo-random shared private key, using the previously exchanged key as seed. Finally, we use the generated random key to encrypt and decrypt messages between the two actors using 128-bit AES ciphers.
\end{abstract}

\section{Introduction}

Our main goal is to be able to communicate secret data over an unsecure communication channel. To do so, we have to encrypt our data before sending it, and decrypt it after recieving it, so that it only appears encrypted on the communication channel.
To perform these encryptions, we want to use a symmetric cipher allowing us to efficiently compute ciphertext from plaintext and vice versa. Before using symmetric encryption, we have to solve the key exchange problem, because we still need to communicate the private key between the two actors over an unsecure channel.
One answer to this first problem is to use public-key cryptography, and most specifically the Diffie-Hellman key exchange protocol. 

\subsection{Diffie-Hellman}

This key exchange scheme allows us to create a shared private key by only communicating public data. It relies on the computational difficulty of computing discrete logarithms. It has been formulated in 1976 by Whitfield Diffie and Martin Hellman. We use it to generate a first shared private key between two actors, Alice and Bob.

\subsection{Blum Blum Shub}

To increase the strength of the exchanged key, we use it as a seed to a cryptographically strong pseudo-random number generator. The chosen stream generator is Blum Blum Shub, which outputs the least significant bit of $x_{{n+1}}$ at each step of the following sequence.

\begin{equation*}
    x_{n+1}=x_{n}^{2} \bmod M
\end{equation*}

with $x_{0} = \text{seed}$ and $M$


\subsection{Advanced Encryption Standard}



\section{Design and Implementation}

\subsection{Diffie-Hellman}
Talk about group used. (https://www.rfc-editor.org/rfc/rfc5114.html)
Naive exponentiation vs pow !
$10^{616}$

\subsection{Blum Blum Shub}

\subsection{Advanced Encryption Standard}


\section{Test Results}
\section{Discussion}
\section{Conclusion}


% ========== References ========== %

\bibliographystyle{unsrtnat}
\bibliography{sources}

\end{document}



